%!TEX root = project.tex

\section*{Project \\ {\small Emerging Tehnologies 2016}}

\noindent
The following document contains the instructions for the project for this module.
In order to set a context for the feedback, first consider the following.

You should be aware that the standard required for submissions at level 8 (fourth year) is higher than at level 7 (third year), which in turn is higher than at level 6 (first and second year).
Significant effort is made to ensure that the standard is fair and consistent across third level institutes, both nationally and internationally.
The standard we set for modules in computing is informed by Quality and Qualifications Ireland's Award Standard for Computing~\cite{qqicomputing}.
Below is a particularly relevant selection of the learning outcomes contained in that document.

\subsubsection*{\small Level 8 (Year 4)}
\emph{
The learner will be able to:
\begin{itemize}
\item describe the limitations of some current computing theories and knowledge.
\item locate and evaluate information through online research.
\item model and design complex computer-based systems in a way that demonstrates comprehension of the trade-off involved in design choices.
\item demonstrate mastery of a complex and specialised area of skills and tools;
\item use and modify advanced skills and tools to conduct closely guided research, professional or advanced technical activity.
\item manage one’s own learning and development, including time management and organisational skills.
\item manage a computer-based project throughout all stages of the lifecycle.
\item apply quality concepts to products and processes of own work.
\end{itemize}
}

\subsubsection*{\small Level 7 (Year 3)}
\emph{
The learner will be able to:
\begin{itemize}
\item integrate concepts learned across a variety of subject areas.
\item identify relevant material on a given topic from available information sources.
\item succinctly present rational and reasoned arguments to a range of audiences.
\item test and confirm the extent to which a computerbased system meets the criteria defined for its current use.
\item develop innovative solutions to pragmatic situations.
\item recognise the suitability of a given solution to a problem.
\item apply knowledge learned in new situations.
\end{itemize}
}
\subsubsection*{\small Level 6 (Years 1 and 2)}
\emph{
The learner will be able to:
\begin{itemize}
\item describe best practices in computing.
\item recognise and apply common best practices.
\item apply knowledge in a practical setting under supervision.
\item demonstrate the capacity to learn new knowledge and skills.
\item use troubleshooting strategies and techniques in correcting a variety of computer hardware and software problems.
\item implement computer based systems solutions to well-defined problems.
\end{itemize}
}



\section*{Project brief}
You are required to develop a single-page (web) application (SPA) written in the programming language Go.

\begin{minted}{go}
Hello, world!
\end{minted}
Note that \mintinline{go}{Hello} is just...~\cite{pythoncollections}.

\section*{Summary}
Best of luck.


